\section{Expansió de l'Islam}\label{expansiuxf3-de-lislam}

\subsection{Motivació}\label{motivaciuxf3}

Moguts pel desig de difondre lla seua nova religió, els musulmans van
aprofitar la feblesa dels imperio veïns per a dur a terme una
\textbf{expansió militar}.

\subsection{Fases}\label{fases}

\begin{itemize}
\itemsep1pt\parskip0pt\parsep0pt
\item
  Segle VII: S'enfronten a l'imperi Bizantí.
\item
  Segle VIII: Assegten fins a Constantinoble, però no van poder
  conquistar-la.
\item
  Segle IX: S'expandeixen per Sud d'Egipte i centre l'Itàlia.
\end{itemize}

\section{Organització}\label{organitzaciuxf3}

\subsection{Etapes}\label{etapes}

\begin{itemize}
\itemsep1pt\parskip0pt\parsep0pt
\item
  \textbf{Primers califes} (632-661): \textbf{xiites} (el califa havia
  d'ésser descendent de Mahoma), \textbf{sunnites} (pensaven que havia
  d'ésser escollit d'entre els millors creients).
\item
  \textbf{Disnastia omeia} (661-750): Instal·lats a Damasc, gran
  expansió per la península Ibèrica.
\item
  \textbf{Dinastia abbassida} (750-1258): Van derrotar els omeies i
  instal·len la capital a Bagdad.
\item
  \textbf{Turcs}: En 1453 van conquistar Constantinoble.
\end{itemize}

\section{Explendor econòmica i
cultural}\label{explendor-econuxf2mica-i-cultural}

\subsection{Trets}\label{trets}

\begin{itemize}
\itemsep1pt\parskip0pt\parsep0pt
\item
  Llengua oficial de l'imperi: \textbf{l'àrab}.
\item
  Els avantatges econòmics i culturals que va gaudir
  l'\textbf{islamització} va afavorir la seua expansió.
\item
  L'imperi es va omplir de \textbf{noves ciutats}, que eren centres
  religiosos, polítics, econòmics i culturals.
\item
  A les ciutats es varen crear: Escoles, bibliloteques i centres de
  traductors.
\end{itemize}

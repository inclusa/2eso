\documentclass{beamer}
\usepackage[utf8]{inputenc}
\usetheme{metropolis}           % Use metropolis theme
\title{Al-Andalus}
\date{\today}
\author{Alfons Rovira}
\institute{http://llagosti.wordpress.com}

\begin{document}

  \maketitle

  \section{0. Introducció}

  \begin{frame}{Introducció}
  
     \begin{itemize}[<+-| alert@+>]
        \item Segle VII: Els mulusmans iniciarn un procés d'expació i conquistaren un gra imperi.
        \item 711 - 1492: La debilitat del regne visigòtic va permetre que els musulmans entraren en la península i es quedaren llarg temps.
        \item El territori ocupat rep el nom d'Al-Andalus, comprén la major part de la Península.
        \item Al llarg del temps aquest territori va estar conquestat pels reis cristians que romanien al nord de la Península.
     \end{itemize}
  \end{frame}

  \section{1. Expansió de l'Islam}
  \begin{frame}{1. Expansió de l'Islam}

Moguts pel desig de difondre lla seua nova religió, els musulmans van aprofitar la feblesa dels imperio veïns per a dur a terme una \textbf{expansió militar}.
   
  \begin{itemize}
  \itemsep1pt\parskip0pt\parsep0pt
  \item Segle VII: S'enfronten a l'imperi Bizantí.
  \item Segle VIII: Assegten fins a Constantinoble, però no van poder conquistar-la.
  \item Segle IX: S'expandeixen per Sud d'Egipte i centre l'Itàlia.

  \end{itemize}

  \end{frame}

  \section{1.1. Etapes}\label{etapes}
  \begin{frame}{1.1. Etapes}

  \begin{itemize}
     \itemsep1pt\parskip0pt\parsep0pt
     \item  \textbf{Primers califes} (632-661): \textbf{xiites} (el califa havia  d'ésser descendent de Mahoma), \textbf{sunnites} (pensaven que havia  d'ésser escollit d'entre els millors creients).
      \item  \textbf{Disnastia omeia} (661-750): Instal·lats a Damasc, gran  expansió per la península Ibèrica.
      \item  \textbf{Dinastia abbassida} (750-1258): Van derrotar els omeies i  instal·len la capital a Bagdad.
      \item  \textbf{Turcs}: En 1453 van conquistar Constantinoble.

   \end{itemize}

   \end{frame}


   \section{2. Explendor econòmica i cultural}\label{explendor-econuxf2mica-i-cultural}

   \begin{frame}{2.Explendor econòmica i cultural}
   	
  \begin{itemize}

    \itemsep1pt\parskip0pt\parsep0pt

    \item  Llengua oficial de l'imperi: \textbf{l'àrab}.
    \item  Els avantatges econòmics i culturals que va gaudir  l'\textbf{islamització} va afavorir la seua expansió.
    \item  L'imperi es va omplir de \textbf{noves ciutats}, que eren centres  religiosos, polítics, econòmics i culturals.
    \item  A les ciutats es varen crear: Escoles, bibliloteques i centres de  traductors.
  \end{itemize}

   \end{frame}

\end{document}

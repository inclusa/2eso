%%%%%%%%%%%%%%%%%%%%%%%%%%%%%%%%%%%%%%%%%
% Beamer Presentation
% LaTeX Template
% Version 1.0 (10/11/12)
%
% This template has been downloaded from:
% http://www.LaTeXTemplates.com
%
% License:
% CC BY-NC-SA 3.0 (http://creativecommons.org/licenses/by-nc-sa/3.0/)
%
%%%%%%%%%%%%%%%%%%%%%%%%%%%%%%%%%%%%%%%%%

%----------------------------------------------------------------------------------------
%PACKAGES AND THEMES
%----------------------------------------------------------------------------------------

\documentclass{beamer}

\usepackage[utf8]{inputenc}

\mode<presentation> {

% The Beamer class comes with a number of default slide themes
% which change the colors and layouts of slides. Below this is a list
% of all the themes, uncomment each in turn to see what they look like.

%\usetheme{default}
%\usetheme{AnnArbor}
%\usetheme{Antibes}
%\usetheme{Bergen}
%\usetheme{Berkeley}
%\usetheme{Berlin}
%\usetheme{Boadilla}
%\usetheme{CambridgeUS}
%\usetheme{Copenhagen}
%\usetheme{Darmstadt}
%\usetheme{Dresden}
%\usetheme{Frankfurt}
%\usetheme{Goettingen}
%\usetheme{Hannover}
%\usetheme{Ilmenau}
%\usetheme{JuanLesPins}
%\usetheme{Luebeck}
\usetheme{Madrid}               % Use Madrid theme
%\usetheme{Malmoe}
%\usetheme{Marburg}
%\usetheme{Montpellier}
%\usetheme{PaloAlto}
%\usetheme{Pittsburgh}
%\usetheme{Rochester}
%\usetheme{Singapore}
%\usetheme{Szeged}
%\usetheme{Warsaw}
%\usetheme{metropolis}           % Use metropolis theme

% As well as themes, the Beamer class has a number of color themes
% for any slide theme. Uncomment each of these in turn to see how it
% changes the colors of your current slide theme.

%\usecolortheme{albatross}
%\usecolortheme{beaver}
%\usecolortheme{beetle}
%\usecolortheme{crane}
%\usecolortheme{dolphin}
%\usecolortheme{dove}
%\usecolortheme{fly}
%\usecolortheme{lily}
%\usecolortheme{orchid}
%\usecolortheme{rose}
%\usecolortheme{seagull}
%\usecolortheme{seahorse}
%\usecolortheme{whale}
%\usecolortheme{wolverine}


%\setbeamertemplate{footline} % To remove the footer line in all slides uncomment this line
%\setbeamertemplate{footline}[page number] % To replace the footer line in all slides with a simple slide count uncomment this line

%\setbeamertemplate{navigation symbols}{} % To remove the navigation symbols from the bottom of all slides uncomment this line
}

\usepackage{graphicx} % Allows including images
\usepackage{booktabs} % Allows the use of \toprule, \midrule and \bottomrule in tables

%----------------------------------------------------------------------------------------
%TITLE PAGE
%--------------------------------------	--------------------------------------------------

\title[Short title]{Al-Andalus} % The short title appears at the bottom of every slide, the full title is only on the title page

\author{Alfons Rovira} % Your name
\institute[Llagostí] % Your institution as it will appear on the bottom of every slide, may be shorthand to save space
{
https://llagosti.wordpress.com \\ % Your institution for the title page
\medskip
}
\date{\today} % Date, can be changed to a custom date

\begin{document}

\begin{frame}
\titlepage % Print the title page as the first slide
\end{frame}


\begin{frame}
\frametitle{Overview} % Table of contents slide, comment this block out to remove it
\tableofcontents % Throughout your presentation, if you choose to use \section{} and \subsection{} commands, these will automatically be printed on this slide as an overview of your presentation
\end{frame}

%----------------------------------------------------------------------------------------
%PRESENTATION SLIDES
%----------------------------------------------------------------------------------------


\section{0. Introducció}

\begin{frame}{Introducció}
  
\begin{itemize}[<+-| alert@+>]
        \item Segle VII: Els mulusmans iniciarn un procés d'expació i conquistaren un gra imperi.
        \item 711 - 1492: La debilitat del regne visigòtic va permetre que els musulmans entraren en la península i es quedaren llarg temps.
        \item El territori ocupat rep el nom d'Al-Andalus, comprén la major part de la Península.
        \item Al llarg del temps aquest territori va estar conquestat pels reis cristians que romanien al nord de la Península.
\end{itemize}

\end{frame}

\section{1. Expansió de l'Islam}

\begin{frame}{1. Expansió de l'Islam}

Moguts pel desig de difondre lla seua nova religió, els musulmans van aprofitar la feblesa dels imperio veïns per a dur a terme una \textbf{expansió militar}.
   
\begin{itemize}

  \itemsep1pt\parskip0pt\parsep0pt
  \item Segle VII: S'enfronten a l'imperi Bizantí.
  \item Segle VIII: Assegten fins a Constantinoble, però no van poder conquistar-la.
  \item Segle IX: S'expandeixen per Sud d'Egipte i centre l'Itàlia.

\end{itemize}

\end{frame}

\subsection{1.1. Etapes}\label{etapes}

\begin{frame}{1.1. Etapes}

\begin{itemize}
     \itemsep1pt\parskip0pt\parsep0pt
     \item  \textbf{Primers califes} (632-661): \textbf{xiites} (el califa havia  d'ésser descendent de Mahoma), \textbf{sunnites} (pensaven que havia  d'ésser escollit d'entre els millors creients).
      \item  \textbf{Disnastia omeia} (661-750): Instal·lats a Damasc, gran  expansió per la península Ibèrica.
      \item  \textbf{Dinastia abbassida} (750-1258): Van derrotar els omeies i  instal·len la capital a Bagdad.
      \item  \textbf{Turcs}: En 1453 van conquistar Constantinoble.
   \end{itemize}

   \end{frame}


\section{2. Explendor econòmica i cultural}\label{explendor-econuxf2mica-i-cultural}

\begin{frame}{2.Explendor econòmica i cultural}
   	
\begin{itemize}

\itemsep1pt\parskip0pt\parsep0pt
    \item  Llengua oficial de l'imperi: \textbf{l'àrab}.
    \item  Els avantatges econòmics i culturals que va gaudir  l'\textbf{islamització} va afavorir la seua expansió.
    \item  L'imperi es va omplir de \textbf{noves ciutats}, que eren centres  religiosos, polítics, econòmics i culturals.
    \item  A les ciutats es varen crear: Escoles, bibliloteques i centres de  traductors.
\end{itemize}

\end{frame}

\end{document}
